\documentclass[12pt, a4paper]{article}


% A pretty common set of packages
\usepackage[margin=2.5cm]{geometry}
\usepackage[T1]{fontenc}
\usepackage{graphicx}
\usepackage{amssymb}
\usepackage{amsmath}
\usepackage{color}
\usepackage{booktabs}
\usepackage{multirow}
\usepackage{engord}
\usepackage{enumitem}
\usepackage{soul}
\usepackage{multicol}
\usepackage{textcomp}
\usepackage{tabularx}
\usepackage{parskip}
\usepackage{setspace}
\usepackage{titlesec}
\usepackage{subfig}
\usepackage{float}
\usepackage{adjustbox}
\usepackage{siunitx,booktabs,caption}
\usepackage{listings}
\usepackage[spanish]{babel}
\usepackage[skip=2pt,font=footnotesize,justification=centering]{caption}
\usepackage{natbib}
 




% Make some additional useful commands
\newcommand{\ie}{\emph{i.e.}\ }
\newcommand{\eg}{\emph{e.g.}\ }
\newcommand{\etal}{\emph{et al}}
\newcommand{\sub}[1]{$_{\textrm{#1}}$}
\newcommand{\super}[1]{$^{\textrm{#1}}$}
\newcommand{\degC}{$^{\circ}$C}
\newcommand{\wig}{$\sim$}
\newcommand{\ord}[1]{\engordnumber{#1}}
\newcommand{\range}[3]{$#1$-$#2\,$#3}
\newcommand{\roughly}[2]{$\sim\!#1\,$#2}
\newcommand{\area}[3]{$#1 \! \times \! #2\,$#3}
\newcommand{\vol}[4]{$#1 \! \times \! #2 \! \times \! #3\,$#4}
\newcommand{\cube}[1]{$#1 \! \times \! #1 \! \times \! #1$}
\newcommand{\figref}[1]{Figure~\ref{#1}}
\newcommand{\eqnref}[1]{Equation~\ref{#1}}
\newcommand{\tableref}[1]{Table~\ref{#1}}
\newcommand{\secref}[1]{Section \ref{#1}}
\newcommand{\XC}{\emph{exchange-correlation}}
\newcommand{\abinit}{\emph{ab initio}}
\newcommand{\Abinit}{\emph{Ab initio}}
\newcommand{\Lonetwo}{L1$_{2}$}
\newcommand{\Dznt}{D0$_{19}$}
\newcommand{\Dtf}{D8$_{5}$}
\newcommand{\Btwo}{B$_{2}$}
\newcommand{\fcc}{\emph{fcc}}
\newcommand{\hcp}{\emph{hcp}}
\newcommand{\bcc}{\emph{bcc}}
\newcommand{\Ang}{{\AA}}
\newcommand{\inverseAng}{{\AA}$^{-1}$}
%\newcommand{\comment}[1]{}
\newcommand{\comment}[1]{\textcolor{red}{[COMMENT: #1]}}
\newcommand{\more}{\textcolor{red}{[MORE]}}
\newcommand{\red}[1]{\textcolor{red}{#1}}

\usepackage{capt-of,graphicx}
\newcounter{pics}

\newcommand\z[2][]{%
  \ifnum\value{pics}=4\par\setcounter{pics}{1}\else\stepcounter{pics}\fi
  \ifhmode\unskip\hfill\fi
  \parbox[t]{.50\textwidth}{%
   \centering\includegraphics[width=\linewidth]{#2}\par
   \ifx\relax#1\relax\else\captionof{figure}{#1}\fi}}

\newcommand{\splitcell}[1]{%
  \begin{tabular}{@{}c@{}}\strut#1\strut\end{tabular}%
}

\begin{document}
\thispagestyle{empty}
	
\begin{figure}[ht]
   \minipage{0.76\textwidth}
		\includegraphics[width=4cm]{img/fi.png}
		\label{EscudoUABC}
   \endminipage
   \minipage{0.32\textwidth}
		\includegraphics[height = 4.5 cm ,width=4.5cm]{img/unam.png}
		\label{EscudoFC}
	\endminipage
\end{figure}
	
\begin{center}
\vspace{0.8cm}
\LARGE
UNIVERSIDAD NACIONAL AUTÓNOMA DE MÉXICO

\vspace{0.8cm}
\LARGE
FACULTAD DE INGENIERÍA

\vspace{1.7cm}	
\Large
\textbf{Filtros}

\vspace{0.2cm}
\LARGE
Reporte de la práctica 8


\vspace{1.3cm}
\normalsize	
PRESENTA \\
\vspace{.3cm}
\large
\textbf{NAVARRETE ZAMORA ALDO YAEL}

\vspace{1.3cm}
\normalsize	
PROFESORA \\
\vspace{.3cm}
\large
\textbf{Norma Reyes Cruz}

\vspace{1.3cm}
\normalsize	
ASIGNATURA \\
\vspace{.3cm}
\large
\textbf{LABORATORIO DE SISTEMAS DE COMUNICACIONES}
\end{center}

\newpage



%--------OBJETIVOS
\section{Objetivos}

El alumno:
\begin{enumerate}
  \item Aprenderá qué es un filtro real y qué tan cerca está su funcionamiento de un filtro ideal.
  \item Conocerá las curvas de respuesta a la frecuencia de cuatro tipos de filtros.
  \item Reconocerá características, zonas o bandas en el diseño de filtros.
\end{enumerate}

% --- Lista de experimentos
\section{Lista de experimentos}
\begin{enumerate}
  \item Obtención de curvas de respuesta a la frecuencia de filtros.
  \item Observar la amplitud a la salida del filtro con relación a la frecuencia de la señal de entrada.
  \item Sabrá calcular la frecuencia de corte de un filtro de primer orden tipo RC.
\end{enumerate}

%---- Lista de equipo

%-----Cuestionario Previo
\section{Cuestionario Previo}

\textbf{1. Investigue y defina qué es un Filtro Eléctrico.}\par 

Un Filtro electrónico es un elemento que deja pasar señales eléctricas a través de él, a una cierta frecuencia o rangos de frecuencia mientras previene el paso de otras, pudiendo modificar tanto su amplitud como su fase. Es un dispositivo que separa, pasa o suprime un grupo de señales de una mezcla de señales.  Pueden ser: analógicos o digitales, los filtros analógicos son aquellos en el que la señal puede tomar cualquier valor dentro de un intervalo, mientras que la señal de los filtros digitales toma solo valores discretos.Un Filtro electrónico es un elemento que deja pasar señales eléctricas a través de él, a una cierta frecuencia o rangos de frecuencia mientras previene el paso de otras, pudiendo modificar tanto su amplitud como su fase. Es un dispositivo que separa, pasa o suprime un grupo de señales de una mezcla de señales.  Pueden ser: analógicos o digitales, los filtros analógicos son aquellos en el que la señal puede tomar cualquier valor dentro de un intervalo, mientras que la señal de los filtros digitales toma solo valores discretos.

\textbf{2. Investigue y anote qué es un filtro ideal, incluya las gráficas en el dominio de la frecuencia de filtros ideales: paso bajas, paso altas, paso banda y supresor de banda.}\par 

A estos filtros se les denominan ideales porque las transiciones entre bandas de paso y bandas rechazadas son bruscas. En la realidad
no existen filtros ideales, si bien es posible diseñar filtros en los que las transiciones
aludidas sean muy aproximadamente bruscas. 


\begin{figure}[H]
  \centering
  \includegraphics[height=5cm]{img/idealfiltros.png}
  \caption{Imagen mostrando el comportamiento de los filtros ideales.}
\end{figure}

\textbf{3. Anote la clasificación de los filtros eléctricos atendiendo al tipo de elementos que utiliza}\par 
Interprétela.}\par

Es similar a la campana de Gauss.

\textbf{4. Investigue y anote la clasificación de los filtros eléctricos según su aproximación.}\par 


\textbf{5. ¿Qué implica el orden del filtro en el espectro de magnitud?}\par 

\textbf{6. ¿Cuál es el criterio más utilizado para determinar la frecuencia de corte de un filtro, y qué representa en potencia y voltaje?}\par 

\textbf{7. La respuesta en frecuencia de un filtro, es la mostrada en la Figura 8.1. Anote 5 bandas, zonas o características del filtro en la figura.}\par 

\textbf{8. ¿Qué es la relación Señal a Ruido y cómo se puede cuantificar?} \par

\textbf{9. Investigue y anote que es el Factor de Ruido y la Cifra de Ruido o Figura de Ruido.} \par

\textbf{10. ¿Qué es el ruido pseudoaleatorio?} \par









\newpage

%------DESARROLLO
\section{Desarrollo de la práctica}

\textbf{1. Conecte el generador de Ruido a la bocina y escuche, anotando en el reporte las impresiones personales acerca del sonido recibido, dónde ha oído antes ese tipo de sonidos y trate de explicar su origen.}

Después de haber conectado un generador análogo a la bocina dentro del laboratorio, mediante el uso de cables caimán
En mi caso, ese sonido emitido por la bocina me recordó a cuando una radio no está sintonizada con el ancho de banda de frecuencias de una estación en específico, entonces existe una presencia de ruido aleatorio. O incluso cuando en la televisión no tenía un canal en sintonía.
Comparamos con los sonidos del metro, el ruido de un estadio, de las cascadas, del ambiente, incluso el ruido de la selva o de la lluvia.
No hay transmisión de ruido, no hay información cuando escuchamos el ruido.

\textbf{2. Ajuste el generador de ruido a su máxima amplitud y conéctelo al osciloscopio, voltímetro y analizador de espectros. Consigne en el reporte las gráficas de tiempo y frecuencia, y el voltaje RMS de la señal. Anote sus comentarios.} \par

Conectando el generador de funciones al multimetro, generando una señal de ruido aleatoria, y haciendo el análisis del espectro de frecuencias se observan las siguientes mediciones


\raggedright
\setlength\parskip{.5cm}
\z[Multímetro]{2mult.jpeg}\z[Osciloscopio]{2osc.jpeg}\z[Analizador de espectros]{2esp.jpeg}

En donde en el multímetro obtenemos una lectura de 500 [mV], en el osciloscopio de la figura 3 una señal singular hecha de ruido generada con el generador de funciones y finlamente en el espectro muchas componentes armónicas sin ningún patrón.

\textbf{3. Conecte el generador de ruido al filtro pasa bajas. Anote en su reporte el oscilograma, el espectro, y sus comentarios.}\par


Podemos observar en las ilustraciones que el filtro pasa bajas efectivamente nos muestra el efecto y el lapso de transición, en donde en el espectro de frecuencias las frecuencias bajas pueden pasar sin ningúna dificultad o problema.

\raggedright
\setlength\parskip{.5cm}
\z[Multímetro]{3mult.jpeg}\z[Osciloscopio]{3osc.jpeg}\z[Analizador de espectros]{3esp.jpeg}

Aquí las lecturas en el documento se obtienen datos ligeramente distintos al anterior, aunque seguimos teniendo una lectura de salida muy baja, ahora la señal de salida no tiene tanta distorsión comparada con la de entrada, y las componentes armónicas ahora empiezan a tomar mejor forma de la que tenían antes, antes no había ningún tipo de patrón en ellas, sin embargo, ahora parece ser que existe mayor amplitud en frecuencias bajas, tal como si actúase como un filtro.

\textbf{4. Alimente simultáneamente una señal senoidal de IKHz, junto con el ruido a la entrada del filtro. Ajuste los valores de ambas señales aproximadamente al mismo valor, con ayuda del analizador de espectros y el voltímetro haga las mediciones necesarias para obtener la relación Señal/Ruido a la entrada del filtro. Consigne el espectro de la señal que alimenta el filtro.}


\raggedright
\setlength\parskip{.5cm}
\z[Generador de funciones]{4gen.jpeg}\z[Osciloscopio]{4osc.jpeg}\z[Analizador de espectros]{4esp.jpeg}

De este punto captamos que para un voltaje de corriente alterna, el multímetro mandó la información de que se está recibiendo una señal de muy pequeña de entrada, del orden de milivolts, esta lectura fué de 0.3922 [V].

\textbf{5. Realice las mediciones necesarias para obtener la relación Señal a Ruido a la salida del filtro.}\par

En la salida del filtro el multimetro arrojó un dato muy pequeño de igual forma, del orden de milivolts, 0.2333 [V].

\textbf{6. Con el mismo filtro, realice las mediciones necesarias para calcular el Factor de Ruido. Anote en su reporte los datos y los resultados. Consigne el espectro de la señal a la salida del filtro.}\par

De acuerdo con el Cuestionario previo:

\begin{gather*}
 F.R = \frac{\frac{S}{R}_{in}}{\frac{S}{R}_{out}}
\end{gather*}

Por lo que

\begin{gather*}
  F.R = \frac{0.3922}{0.2333} = 1.6813
\end{gather*}

\textbf{7. ¿Qué información se puede obtener del Factor de Ruido?} \par

Del inciso anterior podemos deducir u obtener como se menciona en el previo, que, para un Factor de Ruido igual a 1, este factor se encuentra en un estado de equilibrio, si por el contrario, el factor de ruido es mayor a 1, quiere decir que habrá poco ruido, y finalmente, si el factor es menor a 1, habrá bastante ruido.



\textbf{8. Con el mismo filtro, realice las mediciones necesarias para calcular la Cifra de Ruido. Anote en su reporte los datos y los resultados.} \par

Tenemos que para la cifra del ruido, el calculo de acuerdo al previo es:

\begin{gather*}
  C.R = 20log(F.R) \\
  C.R = 20log\big(\frac{\frac{S}{R}_{in}}{\frac{S}{R}_{out}}\big) [dB] \\
  C.R = 4.5129
\end{gather*}

\textbf{9. Genere una señal de ruido pseudoaleatorio. Consigne el oscilograma y el espectro en el reporte con sus comentarios.} \par

Este tipo de señales no son posibles de generarse dentro del laboratorio, entonces también es imposible captar el oscilograma y el espectro dentro del reporte.

\textbf{10.Anote un resumen de lo aprendido en la práctica (por lo menos 5 puntos).} \par

\section{Conclusiones}

En la práctica realizada en el laboratorio de Sistemas de Comunicaciones definimos al ruido, vimos algunas expresiones que nos ayudan a calcular las relaciones que existen entre las señales emitidas por una fuente. Definimos al ruido en distintos escenarios, tales como las cascadas, así como en los estadios, en las televisoras, o incluso en la radio.
Finalmente, analizamos el espectro de frecuencia de dichas señales mediante el generador de funciones, empleando la función de ruido.
Aprendimos de igual forma a clasificar el ruido, y a distinguir entre los distinos rangos de frecuencia qué ruido asociado se encuentra en esta, tal como el ruido rosa, azul, violeta, ultravioleta.

\begin{figure}[H]
  \centering
  \includegraphics[height=12cm]{mapa.jpeg}
  \caption{Mapa mental de lo aprendido.}
\end{figure}

\section{Referencias}
\begin{itemize}
  \item tok.wiki. (s.f.). Ruido pseudoaleatorio Código PNyVer también. https://hmong.es/wiki/Pseudorandom\_noise
\end{itemize}

\end{document}
